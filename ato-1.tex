%\documentclass[a4paper]{scrartcl}
\documentclass[a4paper]{article}
\usepackage{setspace}
\usepackage{url}
\usepackage{mdwlist}
\usepackage{polski}
\usepackage[utf8x]{inputenc}
\usepackage{color}
\usepackage{mathtools}
\usepackage{graphicx}
\usepackage[unicode=true]{hyperref}
\usepackage{multirow}
\usepackage[table]{xcolor}
\usepackage{subfig}
\usepackage{listings}
\usepackage[backgroundcolor=white]{todonotes}
\definecolor{dkgreen}{rgb}{0.2,0.8,0.2}
\definecolor{gray}{rgb}{0.5,0.5,0.5}
\definecolor{mauve}{rgb}{0.58,0,0.82}
\newcommand{\HRule}{\rule{\linewidth}{0.5mm}}
\newcommand{\bsd}{\textbf{FreeBSD}}
\newcommand{\ssh}{\textbf{SSH}}
\newcommand{\rdp}{\textbf{rdesktop}}
\newcommand{\xbin}{\textbf{X}}
\newcommand{\xvnc}{\textbf{Xvnc}}
\newcommand{\wat}{\textbf{wat}}
\newcommand{\volt}{\textbf{volt}}
\lstset{ %
  basicstyle=\ttfamily\footnotesize,
  numbers=left,
  numberstyle=\footnotesize,
  stepnumber=1,
  numbersep=5pt,
  breaklines=true,
  tabsize=2,
  showspaces=false,
  showstringspaces=false,
  frame=single,
  numberstyle=\tiny\color{gray},
  keywordstyle=\color{mauve},
  commentstyle=\color{dkgreen},
  stringstyle=\color{mauve},
}

\begin{document}
\begin{titlepage}

  \begin{center}


    % Upper part of the page
    \includegraphics[width=0.3\textwidth]{logo.jpg}\\[1cm]

    \begin{onehalfspace}
      \textsc{\LARGE Wydział Elektryczny Politechniki Warszawskiej}\\[1.5cm]
    \end{onehalfspace}



    \textsc{Sieci Komputerowe Lab.}\\[0.5cm]

    % Title
    \HRule \\[0.4cm]
    {\huge \bfseries Protokoły graficznych interfejsów użytkownika}\\[0.2cm]
    \HRule \\[1.5cm]

    % Author and supervisor
    \begin{flushleft} \large
      \emph{Autorzy:}\\
      Barnaba \textsc{Turek}\\
      \href{mailto:turekb@volt}{turekb@volt}\\
      Tomasz \textsc{Cudziło}\\
      \href{mailto:cudzilot@volt}{cudzilot@volt}\\
    \end{flushleft}
    \vfill

    % Bottom of the page
    {\large \today}

  \end{center}

\end{titlepage}
\sloppy

\setcounter{tocdepth}{4}
\tableofcontents

\section{Wstęp}
\section{Cel ćwiczenia}
Celem ćwiczenia było zapoznianie się z~systemami oprogramowania umożliwiającymi korzystanie z~graficznych interfejsów użytkownika poprzez sieć.
\section{Badane protokoły i~systemy oprogramowania}
Zapoznaliśmy się z~następującymi protokołami:
\begin{description}
  \item[X11] Protokół powstały dla systemów UN*X\dywiz owych, dostępny także na innych systemach. Korzystaliśmy z~programów \ssh, \xbin,
    a~także różnych skryptów pozwalających wygodnie zarządzać uwierzytelnianiem klientów.
  \item[RDP] Protokół wspierany przez systemy z~rodziny \emph{Windows}. Korzystaliśmy z~klienta \rdp, i~maszyny \wat{} jako serwera.
  \item[VNC] Popularny protokół zdalnego dostępu. Korzystaliśmy z~\xvnc{} jako serwera, a~także z~klientów \textbf{tightVNC} i \textbf{Tiger VNC}.
\end{description}

\section{X11}
\subsection{Schemat poglądowy}
\subsection{Przygotowanie}
\todo{Włączenie Xów, podmontowanie home}
\subsection{Wybór ekranu dla klienta}
\todo{To co ato mówił - że można poza zmienną DISPLAY. format zmiennej. Coś o~zmiennych środowiskowych}
\subsection{Uwierzytelnianie}
\subsubsection{Uwierzytelnianie za pomocą SSH}
\todo{Nie mam pojęcia jak to działa}
\subsubsection{Skrypt xauth-export}
\todo{Jakaś trivia o~tym, jak to działa}
\section{RDP}
\todo{Tutaj możemy chyba dać tylko listing, warto wspomnieć o~haxopcjach linii komend których użyliśmy}
\section{VNC}
\todo{Warto zaznaczyć, że w~przeciwieństwie do windowsów nie łączymy się z~istniejącym ekranem, tylko dostajemy nowy. Że standardowo się forkuje i~trzeba go zabić etc.}

\end{document}

%fragmenty do copypastowania

\begin{figure}[h]
  \centering
  \includegraphics[width=0.8\textwidth]{asd.png}
  \caption{}
\end{figure}

\begin{lstlisting}[caption=Przykład]
  kod
\end{lstlisting}
