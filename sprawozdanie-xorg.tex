\documentclass[a4paper,11pt]{article}
\usepackage{sprawozdanie-ato}

\begin{document}


\begin{titlepage}
  \begin{center}
    % Upper part of the page
    \includegraphics[width=0.3\textwidth]{logo.jpg}\\[1cm]
    \begin{onehalfspace}
      \textsc{\LARGE Politechnika Warszawska}\\[0.5cm]
      \textsc{\Large Wydział Elektryczny}\\[1.5cm]
    \end{onehalfspace}
    \textsc{Sieci Komputerowe --- Laboratorium}\\[0.5cm]

    % Title
    \HRule \\[0.4cm]
    {\huge \bfseries Protokoły graficznych interfejsów użytkownika}\\[0.2cm]
    \HRule \\[1.5cm]

    % Author and supervisor
    \begin{flushleft} \large
      \emph{Autorzy:}\\
      Barnaba \textsc{Turek}\\
      \href{mailto:turekb@volt}{turekb@volt}\\
      Tomasz \textsc{Cudziło}\\
      \href{mailto:cudzilot@volt}{cudzilot@volt}\\
    \end{flushleft}
    \vfill

    % Bottom of the page
    {\large \today}
  \end{center}
\end{titlepage}
\sloppy


\setcounter{tocdepth}{4}
\tableofcontents


\section{Wstęp}

\subsection{Cel ćwiczenia}
Celem ćwiczenia było zapoznanie się z~systemami oprogramowania umożliwiającymi
korzystanie z~graficznych interfejsów użytkownika poprzez sieć.

\subsection{Badane protokoły i~systemy oprogramowania}
Zapoznaliśmy się z~następującymi protokołami:
\begin{description}
  \item[X11] Protokół powstały dla systemów UN*X\dywiz owych, dostępny także na
    innych systemach. Korzystaliśmy z~programów \ssh, \xbin, a~także różnych
    skryptów pozwalających wygodnie zarządzać uwierzytelnianiem klientów.
  \item[RDP] Protokół wspierany przez systemy z~rodziny \emph{Windows}.
    Korzystaliśmy z~klienta \rdp, i~maszyny \wat{} jako serwera.
  \item[VNC] Popularny protokół zdalnego dostępu. Korzystaliśmy z~\xvnc{} jako
    serwera, a~także z~klientów \textbf{tightVNC} i \textbf{Tiger VNC}.
\end{description}


\section{X11}

\subsection{Opis rozwiązania}
Serwer X organizuje komunikację pomiędzy klientami (programami, które chcą
wyświetlać informacje na ekranach) a~sprzętem. Serwer przyjmuje polecenia
rysowania elementów interfejsu od klientów i~w~wyniku tych poleceń rysuje
elementy na ekranie, z~którym jest związany.

Dodatkowo serwer obsługuje urządzenia wejściowe, takie jak klawiatury i~myszki.
Zdarzenia wygenerowane przez te urządzenia przesyła do klientów.

Zastosowany model klienta\dywiz serwera pozwala więc na zdalne uruchamianie
aplikacji z interfejsem graficznych.

W~ramach ćwiczenia uruchomiliśmy serwer \xbin{} na lokalnej maszynie
(przyniesionym laptopie z~systemem Ubuntu lub maszynie laboratoryjnej).
Następnie do uruchomionego serwera podłączaliśmy klientów (programy). Udało nam
się uruchomić klientów graficznych z~maszyny \volt{} oraz własnego laptopa.

\subsection{Przygotowanie}
Po zalogowaniu się na maszynie laboratoryjnej wykonaliśmy następujące polecenia:

\begin{lstlisting}
home -r
startx
\end{lstlisting}

Zamontowanie katalogu domowego, w~którym możemy tworzyć pliki jest niezbędne,
jeśli chcemy skorzystać ze skryptu \texttt{startx}. Wykonuje to za nas skrypt
\texttt{home}. Następnie skrypt \texttt{startx} ustawia zmienną środowiskową
\texttt{XAUTHORITY} na ścieżkę w katalogu domowym użytkownika. Nie jest to
niezbędna akcja, ponieważ jest to domyślna wartość przyjmowana przez serwer:

\begin{lstlisting}
if [ x"$XAUTHORITY" = x ]; then
    XAUTHORITY=$HOME/.Xauthority
    export XAUTHORITY
fi
\end{lstlisting}

\subsubsection{Uruchomienie serwera X}
Zgodnie ze stroną \texttt{man 7 X} plik wskazany przez zmienną
\texttt{XAUTHORITY} zostanie użyty do zapisania klucza, pozwalającego
uwierzytelnić klienta.

W katalogu domowym jest także tworzony plik \texttt{.serverauth.*}, który jest
wykorzystywany do przekazania klucza do serwera.

\subsubsection{Uwierzytelnianie}
Skrypt \texttt{startx} uruchamia serwer \xbin{} korzystający z prostej metody
uwierzytelniania klientów: \texttt{MIT-MAGIC-COOKIE-1}.  Schemat wymiany hasła
zakłada, że zarówno klient jak i serwer znają to samo 16\dywiz sto bajtowe
hasło, podawane serwerowi w czasie jego uruchomienia. Hasło jest pomiędzy nimi
przekazywane za pomocą czystego tekstu. Możliwe jest łatwe podsłuchanie łatwe
przechwycenie hasła.

Poniżej widoczny jest fragment skryptu \texttt{startx} gdzie generowany jest
współdzielony klucz. Następnie zapisywany jest on do pliku, a ścieżka do pliku
dodawana do argumentów wywołania serwera \xbin.

\begin{lstlisting}
if [ -r /dev/urandom ]; then
    mcookie=`dd if=/dev/urandom bs=16 count=1 2>/dev/null | /usr/bin/hexdump -e \\"%08x\\"`
else
    mcookie=`dd if=/dev/random bs=16 count=1 2>/dev/null | /usr/bin/hexdump -e \\"%08x\\"`
fi

[...]

dummy=0

# create a file with auth information for the server. ':0' is a dummy.
xserverauthfile=$HOME/.serverauth.$$
trap "rm -f '$xserverauthfile'" HUP INT QUIT ILL TRAP KILL BUS TERM
xauth -q -f "$xserverauthfile" << EOF
add :$dummy . $mcookie
EOF

serverargs=${serverargs}" -auth "${xserverauthfile}
\end{lstlisting}

Możliwe są także inne metody uwierzytelniania, opisane szczegółowo w \texttt{man
7 Xsecurity}, jednak nie zapoznaliśmy się z nimi bliżej. Niektóre z nich
wyglądają na bezpieczniejsze.

Po uruchomieniu serwera skorzystaliśmy ze skryptu \texttt{xauth-export}, który
wywołuje polecenie \texttt{xauth}, wydobywa z niego klucz, a następnie wgrywa go
na zadaną maszynę za pomocą protokołu \emph{SCP}. Wgraliśmy wygenerowany klucz
na maszynę, na której uruchamialiśmy zdalnego klienta.

Alternatywą do korzystanie z~tego skryptu byłoby skopiowanie wartości z~pliku
wskazywanego przez zmienną \texttt{XAUTHORITY} i~wklejenie ich do odpowiedniego
pliku na maszynie docelowej. Mogliśmy też uruchomić skrypt \texttt{home} bez
argumentów, tak by katalog domowy został zamontowany z~maszyny \volt. Wtedy plik
\texttt{~/.Xauthority} byłby dzielony pomiędzy maszynami.

\subsubsection{Wybór serwera \xbin{} przez klienta}

Ostatnia czynność, którą musieliśmy wykonać przed uruchomieniem klienta (już na
zdalnej maszynie), to ustawienie zmiennej środowiskowej \texttt{DISPLAY}.
Zmienna ta, wg. \texttt{man 7 X} określa, z~którego serwera \xbin ów chcemy
korzystać. Serwer jest identyfikowany przez nazwę hosta i~numer, który przekłada
się bezpośrednio na port, na który otwierane jest połączenie z~serwerem, oraz na
którym ekranie chcemy wyświetlić naszego klienta.

Przykładowo polecenie \texttt{export DISPLAY=volt:0.1} spowoduje, że klienci
\xbin{} będą wyświetlani przez serwer na maszynie \texttt{volt}. Serwer \xbin{}
będzie ich wyświetlał na wyświetlaczu lub grupie monitorów nr \texttt{0} i
ekranie nr \texttt{1}.

Treść \texttt{man 7 X} wspomina, że alternatywą do ustawiania zmiennej
środowiskowej \texttt{DISPLAY} jest podanie klientowi \xbin argumentu
\texttt{-display} z nazwą serwera jako argumentem. Wiele klientów nie obsługuje
tej funkcjonalności.

\subsection{Alternatywa --- \texttt{ssh \dywiz X}}
Program \ssh{} posiada wbudowane rozwiązanie upraszczające proces, przez który
przeszliśmy.

Aby to rozwiązanie zadziałało, przed otwarciem połączenia \ssh{} powłoka,
z~której otwieramy połączenie musi mieć poprawnie ustawione zmienne środowiskowe
związane z~serwerem X (t.j. zmienną \texttt{DISPLAY} i~metodę uwierzytelniania
na tym serwerze).

Jeśli to założenie jest spełnione, to wykonanie polecenia \texttt{ssh} z~opcją
\texttt{-X} automatycznie ustawi na docelowym komputerze poprawną wartość
zmiennej \texttt{DISPLAY} i~poprawną metodę uwierzytelniania (wraz z~kluczem).
Na codzień korzystamy z~tej opcji do wygodnego drukowania plików PDF na serwerze
\volt{} za pomocą programu \emph{Evince}. W~czasie laboratorium nie sprawiła nam
żadnych problemów.


% \section{RDP}
% \todo{Tutaj możemy chyba dać tylko listing, warto wspomnieć o~haxopcjach linii komend których użyliśmy}


% \section{VNC}
% \todo{Warto zaznaczyć, że w~przeciwieństwie do windowsów nie łączymy się z~istniejącym ekranem, tylko dostajemy nowy. Że standardowo się forkuje i~trzeba go zabić etc.}


\end{document}
